\documentclass[
  10pt
, handout
]{beamer}

\usepackage{pgfpages}

% Use T1 Font encoding to support more glyphs
\usepackage[T1]{fontenc}

%\setbeameroption{show notes on second screen}

% Number sections in table of contents
\setbeamertemplate{section in toc}[sections numbered]
% Hide subsections in table of contents
\setcounter{tocdepth}{1}

% Use metropolis beamer theme
\usetheme[
  numbering=fraction  % Page numbering like 3/10
]{metropolis}
\usepackage{appendixnumberbeamer}

\usepackage{booktabs}
\usepackage[scale=2]{ccicons}

\usepackage{pgfplots}
\usepgfplotslibrary{dateplot}

\usepackage{xspace}
\newcommand{\themename}{\textbf{\textsc{metropolis}}\xspace}

\usepackage{graphicx}
\usepackage{listings}

\usepackage{lmodern}

\usepackage{eurosym}
\usepackage{amsmath, amssymb}
\usepackage[binary-units=true]{siunitx}
\DeclareSIUnit{\EUR}{\text{\euro}}

\usepackage{xcolor}
\newcommand\crule[3][black]{\textcolor{#1}{\rule{#2}{#3}}}
\definecolor{aswe-reactive}{cmyk}{1,0.9,0,0}
\definecolor{aswe-proactive}{cmyk}{0.6,0.9,0,0}
\definecolor{aswe-preferences}{cmyk}{0,0.75,1,0}
\definecolor{aswe-data}{cmyk}{0.85,0.1,1,0}

% Content

\title{Voice Controlled User-Interface}
\subtitle{}
\date{April 10, 2019}
\author{Dorian Czichotzki, David Marchi, Daniel Schäfer}

\begin{document}

\maketitle

\begin{frame}{Agenda}
  \tableofcontents[pausesections]
\end{frame}

\section{How does speech recognition work?}

\begin{frame}{How does speech recognition work?}
  AI/ML
\end{frame}

\section{Which applications are suited for control via speech and which requirements are necessary?}

\begin{frame}{Which applications are suited for control via speech and which requirements are necessary?}
  \begin{itemize}
    \item Personal voice assistant
    \item Commands in the car (TODO specify better)
  \end{itemize}
\end{frame}

\section{Advantages and disadvantages of speech controlled dialogs?}

\begin{frame}{Voice controlled dialogs - Advantages}
  Advantages

  \begin{itemize}
    \item Hands free, eyes free
    \item No complex menu interaction
    \item Quick fire and forget command
    \item Could be capable of asking questions
  \end{itemize}
\end{frame}

\begin{frame}{Speech controlled dialogs - Disadvantages}
  Disadvantages

  \begin{itemize}
    \item Hard/impossible to enter non-word content
    \item Low accuracy $\rightarrow$ High chance of not doing what you want
    \item Often limited to a few predefined actions
    \item Often robotic, unnatural commands necessary
  \end{itemize}
\end{frame}

\begin{frame}{Speech controlled dialogs - No visuals}
  No visuals

  \begin{itemize}
    \item Doesn't convey information unless it's talking right at this moment
    \item Can't show multiple things at the same time
    \item Is slower than visuals because you can't skip anything
    \item If you missed something, you missed it
  \end{itemize}
\end{frame}

\section{Guidelines}

\begin{frame}{Guidelines - adapting existing guidelines}
  \begin{itemize}
    \item[{[.]}]<+-> Law of proximity (Objects closer together appear as a single unit)
    \item[{[ ]}]<+-> Law of equality (Similar objects appear as a group)
    \item[{[ ]}]<+-> Law of closedness (Closed structures are more recognizeable)
    \item[{[ ]}]<+-> Law of continuity (Objects in a line or curve appear as single figure)
    \item[{[x]}]<+-> Law of experience
    \item[{[x]}]<+-> Law of same fate \\ (Objects moving in the same way appear as a group)
    \item[{[x]}]<+-> Law of symmetry \\ (Objects that are arranged in symmetry are more easily recognizable)
    \item[{[x]}]<+-> Law of simplicity (The brain simplifies complex shapes)
  \end{itemize}
\end{frame}

\begin{frame}{Shneidermans 8 goldene Regeln (1-4)}
  \begin{enumerate}
    \item<+-> Versuche Konsistenz zu erreichen \\ $\rightarrow$ Gleiche Sprachanfragen sollten gleiche Aktionen hervorrufen
    \item<+-> Biete erfahrenen Benutzern Abkürzungen an \\ $\rightarrow$ Spezielle Wörter, die man einstellen kann, die direkt eine komplexe Aktion aufrufen
    \item<+-> Biete informatives Feedback \\ $\rightarrow$ Gutes TODO
    \item<+-> Dialoge sollten abgeschlossen sein \\ $\rightarrow$ TODO
  \end{enumerate}
\end{frame}

\begin{frame}{Shneidermans 8 goldene Regeln (5-8)}
  \begin{enumerate}
    \setcounter{enumi}{4}
    \item<+-> Biete einfache Fehlerbehandlung \\ $\rightarrow$ TODO
    \item<+-> Biete einfache Rücksetzmöglichkeiten (undo) \\ $\rightarrow$ TODO
    \item<+-> Unterstütze benutzergesteuerten Dialog \\ $\rightarrow$ TODO
    \item<+-> Reduziere die Belastung des Kurzzeitgedächtnisses \\ $\rightarrow$ Kurze Antworten
  \end{enumerate}
\end{frame}

\end{document}
